\documentclass{article}
\usepackage[utf8]{inputenc}
\usepackage{url,color}
\usepackage{biblatex}
\usepackage{graphicx}
\usepackage{siunitx}
\bibliography{bibfile.bib}
\usepackage{amsmath}
\usepackage{enumitem}

\title{American Option Pricer by Three Methods}
\author{Shuangqi Bao, Hong Cao, Ince Yang}
\date{Oct 2023}

\begin{document}

\maketitle
\begin{abstract}
With the use of three unique and sophisticated valuation techniques—the Free Boundary Black-Scholes Model with Finite Difference Method, Binomial Tree Simulation, and Monte Carlo Simulation—this study presents a thorough framework for valuing American options. Because of its complexity and ability to be exercised at any time before expiration, American options present special valuation issues. To present a comprehensive analysis of American option pricing, our study will compare and analyze these three approaches, each of which has advantages and disadvantages of its own.

The early exercise aspect of American options is addressed by the Free Boundary Black-Scholes Model with Finite Difference Method, which provides a reliable method for managing a range of market circumstances. The applicability of the Binomial Tree Simulation in real-world circumstances is investigated. It is well-known for its adaptability and efficiency in showing path-dependent choices. The application of the Monte Carlo Simulation to American options is also investigated. The simulation is praised for its adaptability and capacity to take into account intricate market conditions.
\end{abstract}

\section{Introduction}
The valuation of American options, distinct for allowing holders the option to exercise anytime before expiration, is a vital and complex aspect of financial derivatives during trading. These options are particularly relevant in scenarios where early execution can significantly enhance profits or mitigate risks. Developing precise valuation methods for American options is essential for investors and financial institutions, enabling them to make informed decisions in dynamic and often unpredictable markets.

This paper offers a detailed analysis of American option pricing through the lens of three sophisticated valuation techniques. Each method is uniquely suited to capture the complexities inherent in American options:

Binomial Tree Simulation: This method models the potential future movements of an asset's price using a discrete-time framework focusing on the idea of dynamic programming. It is highly versatile, allowing for adjustments to reflect dividends and other market variables. The binomial tree is particularly effective in visualizing and analyzing the path-dependent nature of options, making it valuable for educational and strategic planning purposes. However, its accuracy depends heavily on the number of time steps chosen, and it can become computationally demanding for large trees.

Monte Carlo Simulation \cite{Boyle1977May}: This stochastic method uses random sampling to simulate the underlying asset's price paths, providing a probabilistic approach to option pricing. It is extremely flexible and can easily incorporate a variety of complex market factors, including stochastic volatility and varying interest rates. Monte Carlo simulation is especially useful for pricing path-dependent derivatives, like American options. However, its accuracy is contingent on the number of simulations run, and it can be computationally intensive, especially for options with a large number of possible exercise dates.

Free Boundary Black-Scholes Model with Finite Difference Method \cite{Cen2011May}: This approach extends the classic Black-Scholes model to accommodate the early exercise feature of American options. The finite difference method is used to numerically solve the resulting partial differential equations. This model is particularly adept at handling a range of market conditions and can be adapted to various underlying asset dynamics. However, its computational intensity and the complexity of implementing boundary conditions can be challenging.

In addition to exploring these models, the paper will address significant considerations in American option pricing. This includes the resolution of the free boundary problem in the Black-Scholes framework, a challenge that is critical to accurately capturing the essence of American options. By integrating these considerations, our analysis aims to enhance the practical applicability and realism of our pricing models, providing a comprehensive toolkit for practitioners and researchers in the field of financial derivatives. The folloing paper is organized as follows: section two gives a comprehensive explanation of how can those three different methods be applied to solving American option pricing. Section three gives quantitative results of three methods with real examples. Section four concludes the whole paper and proposes potential improvements. 


\section{Methodology}

\subsection{Binomial Tree Simulation}
The binomial pricing model monitors how the essential underlying factors of an option change over time in discrete intervals. To achieve this, it employs a binomial lattice or tree, spanning several time steps between the valuation and expiration dates. At each node within this lattice, a potential price for the underlying asset at a specific moment is represented.

The valuation process unfolds in iterations, commencing from the final nodes (those accessible at the expiration date) and then retracing its steps backward through the tree, ultimately reaching the initial node (valuation date). At each stage, the calculated value reflects the option's worth at that particular timestep.

This method of option valuation, as explained, encompasses a three-stage procedure.
\begin{enumerate}
    \item Binomial Tree Generation \\
    In each step, it is assumed that the underlying instrument's price will either increase or decrease by a specific factor (denoted as "u" or "d") during each step of the tree. Here, it's defined that $u \geq 1$ and $0 < d \leq 1$. So, if $S$ represents the current price, then in the next period, the price will be either $Su$ or $Sd$, where $Su$ is calculated as $S$ multiplied by $u$, and $Sd$ is calculated as $S$ multiplied by $d$.
    
    The values of the "up" and "down" factors are determined based on the underlying volatility ($\sigma$) and the duration of each time step ($\Delta t$), which is measured in years using the day count convention of the underlying instrument. To satisfy the condition that the variance of the natural logarithm of the price is $\sigma^2 \Delta t$, we have the following formulas:
    
    $$u = e^{\sigma \sqrt{\Delta t}}$$
    $$d = e^{-\sigma \sqrt{\Delta t}}$$
    
    The above expressions describe the original Cox, Ross, and Rubinstein (CRR) method. \cite{Cox1979Sep}
    
    
    \item Calculate option value at each final node

    At each terminal (expiration date) node of the tree, the option's intrinsic value is:
    
    For a call option: $\max\left[(S_n - K), 0\right]$ \\
    For a put option: $\max\left[(K - S_n), 0\right]$
    
    Here, $K$ is the strike price, and $S_n$ is the spot price of the underlying at the $n$th period.

    \item Sequential calculation of the option value at each preceding node

    By working backward, the expected value at each node is determined by using the risk neutrality assumption which states today's fair price of a derivative is equal to the expected value of its future payoff discounted by the risk-free rate. 

    \begin{align*}
        C_{t-\Delta t,i} = e^{r\Delta t}(p C_{t,i} + (1 - p)C_{t,i+1})
    \end{align*}
    where $C_{t,i}$ is the option's value of the $i^{th}$ node at time $t$, $p = \frac{e^{(r-q)\Delta t} - d}{u - d}$ is by matching the first two moments of Geometric Brownian motion, $\Delta t$ is time period, $r$ is the risk-free rate. Importantly, for the American Option, since the option has the right to be executed at any time, the value at each node is:\\
    For a call option: $\max\left[(S_n - K), C_{t-\Delta t,i}\right]$ \\
    For a put option: $\max\left[(K - S_n), C_{t-\Delta t,i}\right]$

    
\end{enumerate}

\subsection{Monte-Carlo Simulation}

Monte Carlo simulation is a fundamental tool in financial modeling and quantitative research, known for its durability and adaptability in managing complex and unpredictable systems. 

Primary methods for pricing the American option are binomial tree and finite difference method to solve free boundary partial differential equation. However, the drawback is those methods are only able to handle one or two uncertainties. The computational cost may increase exponentially with more variables. 

Monte Carlo simulation, introduced by Stanislaw Ulam in the 1940s, stands out as a widely embraced numerical method for option pricing. This technique involves simulating various paths for asset prices. In the context of n-dimensional problems, Monte Carlo methods offer faster convergence to solutions, demand less memory, and are simpler to implement in programming.

Here are the general steps of the Monte Carlo Method  to evaluate financial derivatives:
\begin{enumerate}
    \item Generate simulations of the underlying state variables, such as asset prices and interest rates, over the specified period using the risk-neutral measure.
    \item Calculate the present value of cash flows for option in each simulated path.
    \item Compute the average present value across all simulated paths. By the Law of Large Numbers (LLN), the average value from our approximate projection has no significant difference from the true results.
\end{enumerate}

\subsubsection{Generate Simulation Stock Paths}

 Consider the Geometric Brownian Motion Model:
    \begin{align*}
         dS_t = \mu S_t dt + \sigma S_t dW 
    \end{align*}
    Then, by the Euler scheme, we can easily compute the iterative step
    \begin{align*}
         dS_{t+dt} = (1+\mu dt) S_t  + \sigma S_t  \epsilon \sqrt{dt} 
    \end{align*}
    where $\epsilon$ is a random variable that follows Standard Normal Distribution.
    
    This concept is straightforward to understand and apply, but the marginal distribution of each value follows a normal distribution instead of a log-normal one. Although reducing the discretization step (DT) can minimize errors, it becomes time-consuming. For this particular scenario, to eliminate the discretization error entirely, we can employ Ito's Lemma directly. However, it's essential to note that this approach may not be universally applicable. In more complex situations with intricate differential equations, generating the entire simple path may be necessary.
    
    By using Ito's Lemma to the GBM we can get
    \begin{align*}
         d\log S_t = \left(\mu - \frac{1}{2}\sigma^2\right) dt + \sigma dW_t
    \end{align*}
    By taking the integral of BHS from 0 to $T$ we get
    \begin{align*}
         S_t = S_0 \exp\left((\mu - \frac{\sigma^2}{2}) dt + \sigma \int_{0}^{t} dW(\tau)\right)
    \end{align*}
    Then we discretise time $t$ into $dt$:
    \begin{align*}
         S_{t+dt} = S_t \exp\left((\mu - \frac{\sigma^2}{2}) dt + \sigma \epsilon \sqrt{dt} \right)
    \end{align*}
    where $\epsilon \sim \mathcal{N}(0, 1)$

    From this equation, we can easily write a MATLAB code to simulate the stock price.

    \begin{figure}
        \centering
        \includegraphics[width=0.9\linewidth]{Monte Carlo simulation.png}
        \caption{Simulation Stock Paths}
        \label{fig:enter-label}
    \end{figure}

\subsubsection{Least Square Monte Carlo}
However, a challenge arises when applying Monte Carlo valuation to American options. Recall for American Options at any exercise time, the holder has the privilege to exercise at any time before the maturity date. 

Therefore, the optimal exercise strategy is fundamentally established based on the conditional expectation of the payoff derived from the decision to continue holding the option. But the conditional expectation we need for every time point should acquired from Monte Carlo. This makes Monte Carlo simulation for an American option have a “Monte Carlo over Monte Carlo” feature, which is computationally expensive. Here we will use least square Monte Carlo(LSM).

Thanks to Longstaff and Schwartz, the key insight underlying our approach is that this conditional expectation can be estimated from the cross-sectional information in the simulation by using least squares.\cite{LS} 

In detail, we conduct a regression analysis using the actual payoffs realized after deciding to continue holding the option, and we regress these payoffs on functions representing the values of the state variables. The resulting fitted values from these regressions serve as direct estimates of the conditional expectation. By estimating this function for each exercise date, we acquire a comprehensive description of the optimal exercise strategy along each path. With this detailed specification, accurate valuation of American options through simulation becomes possible. 

Also thanks to Tsitsiklis and Van Roy, they provided convergence results and error bounds that establish that such methods are viable, as long as state sampling is carried out by simulating the natural distribution of the underlying state process. This provides theoretical support for the apparent effectiveness of this particular form of state sampling.\cite{TJB}

\subsubsection{Perform Least Square Regression on MATLAB}
The least-squares approximate solution of $Ax$ = $y$ is given by
\begin{align*}
    x_{ls} = (A^TA)^{-1}A^Ty
\end{align*}
This is the unique $x \in R^n$ that minimizes $\lVert Ax - y \rVert$.

Although there are several ways to solve $x_{ls}$, the easiest way is to use the backslash operator:
\begin{align*}
    x_{ls} = A \backslash y
\end{align*}
If A is square (and invertible), the backslash operator just solves the linear equations, i.e., it computes $A^-1 y$. If $A$ is not full rank, then $A \backslash b$ will generate an error message, and then a least-squares solution will be returned.

\subsubsection{The LSM Algorithm}
Our method starts with N simulation stock paths 
\begin{align*}
    (S^k_n, t_n) \text{ for } 1 \leq k \leq N \text{ , } t_n = ndt
\end{align*}
Valuation is through a backward induction process. Suppose we've already known $P^k_{n+1} = P(S^k_{n+1},t_{n+1})$. For points $(S^k_n, t_n)$ set $X = S^k_n$ the current option value we get the holding value:
\begin{align*}
    Y = e^{-rdt} F(S^k_{n+1}, t_{n+1})
\end{align*}

For simplicity purposes, we choose basis functions to be $1, X, X^2, X^3$ to perform the least square regression. (Other types of basis functions include the Hermite, Legendre, Chebyshev, Gegenbauer, and Jacobi polynomials.)

i.e. approximate $Y^k$ by a least square based on the polynomial of $X$, which is the holding value of the American option.

To implement this method, we use only in-the-money paths in the estimation since the exercise decision is only relevant when the option is in the money. By focusing on the in-the-money paths, we limit the region over which the conditional expectation must be estimated, and far fewer basis functions are needed to obtain an accurate approximation to the conditional expectation function.


After estimating the conditional expectation function at time $t_{n}$ we can assess the optimality of early exercise at a time 
$t_n$ for an in-the-money path by comparing the immediate exercise value with the expected value, repeating this process for each in-the-money path. Once the exercise decision is determined, we can then approximate the value at time $t_{n-1}$

\subsubsection{About Convergence Result and Error Bound of LSM}
In our research, we didn't discuss the  Convergence Result and Error Bound of the LSM method. However, if you are really interested in these topics. They have been detailly discussed in these reference essays.\cite{LS}\cite{TJB}

\subsection{Free Boundary Black-Scholes with Finite Difference Method}
\subsubsection{Crank-Nicolson Discretization}
Partial Differential Equations (PDEs) are considered a common approach for pricing American options. Compared to the previous two methods, it has a huge run-time advantage. One of the most famous methods of numerically solving PDE is the Crank-Nicolson Method. \\

First, we introduce the Black-Scholes Model. The Black-Scholes Model was used on European option pricing problems. 
$$\frac{\partial V}{\partial t} + \frac{1}{2} \sigma^2 S^2 \frac{\partial^2 V}{\partial S^2} + rS \frac{\partial V}{\partial S} - rV = 0$$

 In the American Option pricing problem, we can rewrite the original Black-Scholes PDE like this $$\frac{\partial U(t,x)}{\partial t}  = \frac{D(t,x)}{2}+ \frac{D(t,x)}{2}  \frac{\partial^2 U(t,x)}{\partial x^2} + F(t,x) \frac{\partial U(t,x)}{\partial x} + R(t,x) $$ 
 
First, by the finite difference method, we construct a 2-dimensional time-space grid. We define the equally spaced grid of $N$ points for the variables $x = log(S/Spot)$ with a spanning region $\pm av\sqrt{T}$. Here the time dimension is also equally spaced with $N$ points.\\

Then by the Crank-Nicolson Method, we discretize the original continuous model to :
$$\frac{U^{(t+1)} - U^{(t)}}{\tau} = \frac{1}{2}( \hat{L}U^{(t+1)} + \hat{L}U^{(t)})$$

where the tridiagonal operator $\hat{L}$ is 

\begin{equation}
  \hat L=\begin{cases}
    l_{j,j+1} = \frac{1}{2h^2}D(x_j)+\frac{1}{2h}F(x_j)\\
    l_{j,j-1} = \frac{1}{2h^2}D(x_j)-\frac{1}{2h}F(x_j)\\
   l_{j,j} = -\frac{1}{h^2}D(x_j)+R(x_j).
  \end{cases}
\end{equation}

We can also write it in a more compact form:
$$\hat{L}U = l_{j,j+1}u_{j+1} + l_{j,j}U_{j} +l_{j,j-1}u_{j-1}$$

Now we have obtained the tridiagonal system we want to solve. As Crank-Nicolson is an implicit method, it ensures the convergence of the solution and avoids the problem of huge oscillations. \\

To make it easier to deploy in our program, we have to make one more transformation. First, we let:
$$U^{t+1} = 2\Tilde{U} - U^{(t)}$$
while implies:
$$(1-\frac{\tau}{2} \hat L)\Tilde{U} = U^{(t)}$$
The explicit expansion of the formula becomes:
$$a_j\Tilde{u}_{j+1} + b_j\Tilde{u}_{j} + c_j\Tilde{u}_{j-1} = u^{(t)}_j$$
with
$$a_j = -\frac{\tau}{4h^2}D(x_j) - \frac{\tau}{4h}F(x_j)$$
$$b_j = 1 + \frac{\tau}{2h^2}D(x_j) - \frac{\tau}{2h}R(x_j)$$
$$c_j = - \frac{\tau}{4h^2}D(x_j) + \frac{\tau}{4h}F(x_j)$$
where $\tau$ is the time step size.
\subsubsection{Free Boundary Problem and "Value and Slope" Conditions}
In the last subsection, we used the finite difference method to construct the 2-dimensional grid and we used the Crank-Nicolson method to discretize the continuous time model. However, we are still a few steps away from solving the pricing problem. As American options generally have the property of early exercise, the Black-Scholes PDE we want to solve is a free boundary problem(FB). In general, we can convert free boundary problems to linear complementary problems. Here we can use a simpler method. We impose two "Value and slope" boundary conditions: $$\Tilde{u}_j - \Tilde{u}_{j-1} = E_j - E_{j-1}$$ $$\Tilde{u}_{j-1} = E_{j-1}$$

where $\Tilde{u_j}$ represents the option holding value at the space grid$j$. Similarly, $E_j$ represents the intrinsic value at the space grid $j$. The second equation is natural since we want to exercise the options we have when the intrinsic value is greater than the holding value. The first equation is less obvious but gives us huge advantages for ensuring the smoothness of the optimal stopping boundary.\\

By solving the tridiagonal system under the "Value and slope" boundary with the Thomas algorithm, we can solve for a list of points that are close to the optimal exercising boundary. However, this is still an issue we need to solve. In the beginning, since time is continuous, we used the Crank-Nicolson discretization scheme to make it numerically computable. However, one problem raised is what if the optimal stopping time boundary does not touch any points on the grids which made the "Value and Slope" condition never meet. In fact, most of the time the optimal stopping boundary does not cross the nodes at all. To solve this problem, we use the idea of shrinking the interval where the optimal stopping boundary exists. \\

To iteratively capture the desired stopping time, we need to impose two concepts: Deep in-the-money end and Deep out-of-money end. The deep-in-the-money end here represents the nodes on the grid that $\Tilde{u_j} - E_j$ is a significant negative number, while the deep out-the-money end represents the nodes on the grid that $\Tilde{u_k} - E_k$ is a significant positive number. The spirit here is similar to the bisection method. As time is continuous, we can guarantee that the $\Tilde{u}_{t'} = E_j$ is satisfied with some $t'$(even though is not one of our time dimensional nodes). Without loss of generality, we assume that $j>k$ Then by iteratively choose $j',k'\in \mathbf{R}$ such that $|j'-k'| < |j - k|$ until there are no such $j',k'$ such that $j'<k$ and satisfies $$\Tilde{u_k'} - E_k' > 0$$ and 
$$\Tilde{u_j'} - E_j' <0 $$ Thus we can conclude that we have found the points that are the lower and upper bound of the optimal stopping boundary of the grid. In the computer program, we also updated the values of all points outside the boundary to $$u_j = E_j$$

Finally, we interpolate all of the $u_{ij}$ with $j$ being the closest grid point (either from left or right but keeping the consistency of picking sides) to construct an accurate estimation of the optimal stopping boundary. Thus, by plugging $t_0$ in the optimal boundary expression, we can get the desired American option price. \\

To sum up, to solve the American Put Option pricing problem by the PDE method, we need to solve Black-Scholes PDE as a free boundary problem. By using the Crank-Nicolson Method, we discretize the continuous time model to a tridiagonal matrix system. By imposing the "Value-Slope Condition", we can make the free boundary problem numerically solvable. In the end, by using the Thomas algorithm to solve the tridiagonal systems and get the boundary estimation points, we interpolate them and find the desired American option price at the initial time. 
\subsubsection{Performance Improvement Strategies}
Under a real quantitative trading context, the algorithm we use needs to be both correct and fast. Here we propose a few ideas to speed up the PDE method.\\

1. We notice that the construction of the grid can affect the order of convergence of the discretized function. In this project, we set the equally spaced grid of $N$ points for the variables $x = log(S/Spot)$ with a spanning region $\pm av\sqrt{T}$ and also equally spaced time grids. We believe that by tuning the ratio between space and time$\frac{x}{\tau}$, and the initial step size and following step size such as $\frac{\tau_0}{\tau}$, we may able to generate a model with a higher order of convergence. \\

2. In this project, our models are implemented in Python/MatLab. However, in future research, we will implement the models in programming languages with run-time advantages such as C++.

\section{Quantitative Results}
Pricing results:
\begin{itemize}[label={}]
    \item Binomial: 11.4527 
    \item Monte Carlo: 11.452917
    \item PDE with Free Boundary: 11.4465
    \item Relative Error is less than 0.0005
\end{itemize}



\section{Conclusions}
In conclusion, while all three methods - Binomial Tree, Monte Carlo, and Finite Difference - accurately predict prices with minimal errors and very close results, each has distinct advantages and drawbacks. The Binomial Tree Method stands out for its ease of implementation, making it highly accessible for those new to financial modeling. On the other hand, the Monte Carlo Method shines with its versatility, offering the greatest potential to accommodate a wide range of variations and complex scenarios. This adaptability is particularly valuable in dynamic financial markets. Lastly, the Finite Difference Method is noteworthy for its speed, delivering results more rapidly than its counterparts. This efficiency is a significant asset in environments where time is of the essence. Therefore, the choice among these methods depends on the specific requirements of the task at hand, balancing factors such as ease of use, adaptability, and speed. 


\printbibliography

\end{document}
